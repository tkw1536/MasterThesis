% !TEX root = ../thesis.tex
\section{Utilizing the Math-In-The-Middle Approach}\label{sec:mitm}

Recall that we want to make use of the Math-In-The-Middle approach to enable communication between different systems. 
The reasons for choosing the Math-In-The-Middle approach are first re-iterated, to then better explain the utilization of it. 

There are two common approaches to make knowledge from one mathematical system $B$ available in another system $A$:
\begin{itemize}
  \item Accessing $B$s API from $A$ and translating the results, and
  \item Copying all knowledge available in $B$ and integrating it into $A$. 
\end{itemize}

If one uses the first approach, one has to rely on the $B$ being constantly available. 
Furthermore, because $A$ suddenly has to read knowledge from $B$, it has to be made aware of how the knowledge in $B$ is represented. 
Thus if $B$ is updated, or new features are added, the integration is likely to break. 

The second approach is also subject to this problem, but to a much bigger extent. 
Not only does $A$ have to be made aware of the knowledge structure of $B$, it is also required for developers to find a way to integrate the knowledge as a whole into the system. 
If $B$ is updated, this approach commonly fails to scale, as it might become necessary to re-integrate the entire set of knowledge. 
This can lead to a situation where $A$ entirely absorbs $B$, and future updates are only made within $B$. 
While this can be a good thing, it also gives less value to $B$ and places a burden on the developers of $A$. 

\subsection{The Math-In-The-Middle Approach}\label{sec:mitm:mitm}

In \mmt, we commonly choose this second approach -- that is we make use of an Import/Export metaphor. 
We have already explain this in Section~\ref{sec:intro:external} and we briefly reiterate it here to allow the reader to better understand the Math-In-The-Middle approach. 

We commonly create an exporter from the external system that turns the knowledge representation of the external system directly into \omdocmmt. 
This knowledge is then stored by the \mmt\ system in a set of XML files representing \omdocmmt. 
These XML files are then used by \mmt\ just like any other concrete theory is used. 

Both approaches are one-way and not scaleable. 
If one wishes to reverse the integration, one has to build a completely new integration. 
Furthermore, to extend this model to $n$ different systems that all make their knowledge available to each other, one has to build connections between any two. 
This results in $n (n - 1) = O(n^2)$ total connections having to be built. 
Recall that this can be seen in Figure~\ref{fig:classicalconnect}. 
This does not scale well with a large number of systems. 

Furthermore recall that we model the true, underlying mathematics inside of \mmt\ along with the knowledge contained in the systems themselves. Making use of theory graphs, this allows us to model translation between this knowledge and the different systems using views. 

We call this approach the Math-In-The-Middle approach (or MiTM for short) and it directly solves the scalability problem mentioned above. 
When adding a new system, we no longer need to build translations between the new system and all other systems, instead we only need to model the knowledge with the help of the Math-In-The-Middle ontology. 
Thus for a set of $n$ systems, we only need to implement $2n$ translations, one from and one to the underlying mathematical knowledge for each of the systems. 
Recall that this can be seen in Figure~\ref{fig:mitmconnect}. 

\subsection{Systems in the OpenDreamKit Project}\label{sec:mitm:systems}

Recall that the OpenDreamKit \cite{OpenDreamKit:on} is an EU-funded project that aims to create Virtual Research Environments enabling mathematicians to make efficient use of existing Open-Source mathematical software systems.
We briefly introduce the main systems used in the OpenDreamKit project, as these are the use-case for cross-system communication. 

For more details, as well as a comprehensive list of other projects in the project and in particular the Math-In-The-Middle approach we refer the interested reader to the OpenDreamKit project proposal \cite{ODKproposal:on} as well as reports on the Math-In-The-Middle approach \cite{DehKohKon:iop16}.

\begin{description}
  \item[LMFDB]
    The \lmfdb\ \cite{lmfdb} project is a mathematical database. 
    It is implemented in Python with a MongoDB backend. 
    The project contains several thousand L-Functions and curves as well as their properties.
    These mathematical objects are stored inside of several sub-databases, for example the elliptic curves database or the galois groups database
    \footnote{We will go into more details on this in Section~\ref{sec:vt:lmfdb} where \lmfdb\ will serve as one of our main examples for virtual theories. }. 
  \item[GAP]
    The GAP system \cite{gap} is a computer algebra system. 
    It is written in C. 
    The abbreviation stands for ``Groups, Algorithms and Programming'' and the system focuses on content from group theory. 
  \item[SageMath]
    SageMath \cite{sagemath} is a project which has built a mathematical knowledge system written in Python. 
    It combines multiple pre-existing projects and allows them to interact via a common interface using Python. 
  \item[OEIS]
    The On-Line Encyclopedia of Integer Sequences \cite{oeis} is a website that collects and indexes sequences of integers. 
    Each sequence is represented by a text file containing informally specified key-value pairs. 
    The entire dataset encompasses about 280000 hand curated sequences. 
\end{description}

\subsection{The Math-In-The-Middle Ontology}\label{sec:mitm:ontology}

\begin{figure}
  \begin{center}
    \begingroup
    \pgfdeclarelayer{background}
    \pgfdeclarelayer{foreground}
    \pgfsetlayers{background,foreground}
    
    \resizebox{\textwidth}{0.90\textwidth}{
      \begin{tikzpicture}[xscale=3.0,yscale=2.0]
        \begin{pgfonlayer}{foreground}
          \tikzstyle{withshadow}=[draw,drop shadow={opacity=.5},fill=white]
          \tikzstyle{database} = [cylinder,cylinder uses custom fill,
            cylinder body fill=yellow!50,cylinder end fill=yellow!50,
            shape border rotate=90,
            aspect=0.25,draw]

          \tikzstyle{human} = [red,dashed,thick]
          \tikzstyle{machine} = [green,dashed,thick]
          
          % Foundations
          \node[thy,dashed,align=center]  (compf) at (0,6.7) {Computational\\Foundations};
          
          \node[thy,dashed,align=center]  (pf) at (-.9,5.7) {Python\\Foundations};
          \node[thy,align=center]  (mf) at (.16,5.3) {Mathematical\\Foundations};
          \node[thy,dashed,align=center]  (cf) at (1,5.7) {C\textsuperscript{++}\\Foundations};

          \node[thy,dashed,align=center]  (sf) at (-0.9,4.8) {SAGE};
          \node[thy,dashed,align=center]  (gf) at (1,4.8) {GAP};

          \draw[include] (compf) -- (pf);
          \draw[includeleft] (compf) -- (cf);
          \draw[include] (pf) -- (sf);
          \draw[includeleft] (cf) -- (gf);
          
          % Model for Elliptic Curves
          \node[thy] (kec) at (0,3) {EC};
          \node[thy,minimum height=.4cm] (kl) at (0,4) {\ldots};
          
          % System-near models (Sage EC, GAP EC, LMFB EC)
          \node[thy] (sec) at (-1,2.6) {Sage EC};
          \node[thy,minimum height=.4cm] (sl) at (-1,3.5) {\ldots};

          \node[thy] (gec) at (1,2.6) {GAP EC};
          \node[thy,minimum height=.4cm] (gl) at (1,3.5) {\ldots};

          \node[thy] (lec) at (-.2,1.2) {LMFDB EC};
          \node[thy,minimum height=.4cm] (ll) at (0.7,1.2) {\ldots};
          
          % Sage System
          \node (sc) at (-2.3,3.5) {SAGE};
          \node[draw] (salg) at (-2.3,3.0) {Algorithms};
          \node[database,dashed] (sdb) at (-2.3,2.5) {DB?};
          \node[draw] (skr) at (-2.3,2.1) {Representation};
          
          % GAP System
          \node (gc) at (2.3,3.5) {GAP};
          \node[draw] (galg) at (2.3,3.0) {Algorithms};
          \node[database,dashed] (gdb) at (2.3,2.5) {DB?};
          \node[draw] (gkr) at (2.3,2.1) {Representation};
          
          % LMFDB System
          \node (lmfdb) at (0,0) {LMFDB};
          \node[database] (ldb) at (.6,-.4) {MongoDB};
          \node[draw] (knowls) at (-.6,-.4) {Knowledge};
          
          % The clouds
          \begin{pgfonlayer}{background}
            \node[draw,cloud,fit=(sec) (sl),aspect=.4,inner sep=-3pt,withshadow,purple!30] (st) {};
            \node[draw,cloud,fit=(gec) (gl),aspect=.4,inner sep=-4pt,withshadow,purple!30] (gt) {};
            \node[draw,cloud,fit=(kec) (kl),aspect=.4,inner sep=0pt,withshadow,blue!30] (kt) {};
            \node[draw,cloud,fit=(lec) (ll),aspect=2.5,inner sep=-7pt,withshadow,purple!30] (lt) {};
          \end{pgfonlayer}
          
          % The system boxes
          \begin{pgfonlayer}{background}
            \node[draw=none,fill=green!30,rounded corners=1cm,fit=(mf) (compf) (pf) (cf) (sf) (gf),inner sep=1pt] {};
            \node[draw,withshadow,fit=(sc) (skr) (sdb),inner sep=1pt] {};
            \node[draw,withshadow,fit=(gc) (gkr) (gdb),inner sep=1pt] {};
            \node[draw,withshadow,fit=(lmfdb) (ldb) (knowls),inner sep=1pt] {};
          \end{pgfonlayer}
          
          % View, Includes metas
          \draw[view] (kec) -- (sec);
          \draw[view] (kec) -- (gec);
          \draw[view] (kec) -- (lec);
          \draw[include] (kec) -- (kl);
          \draw[include] (gec) -- (gl);
          \draw[include] (sec) -- (sl);
          \draw[include] (lec) -- (ll);
          \draw[view] (kl) -- (sl);
          \draw[view] (kl) -- (gl);
          \draw[view] (kl) to[bend left=5] (ll);

          \draw[meta] (mf)  to [bend right=10] (st);
          \draw[meta] (sf) -- (st);
          \draw[meta] (mf)  to [bend left=10] (gt);
          \draw[meta] (gf) -- (gt);
          \draw[meta] (mf) -- (kt);
          \draw[meta] (compf) to[bend right=15] (kt);

          \draw[human,->] (skr) -- node[above]{\scriptsize edit} (st);
          \draw[human,->] (gkr) -- node[above]{\scriptsize edit} (gt);
          \draw[human,->] (knowls) -- node[left,near end]{\scriptsize edit} (lt);

          \draw[human,->] (st) to[bend left=20] node[below]{\scriptsize refactor} (kt);
          \draw[human,->] (gt) to[bend right=20] node[below]{\scriptsize refactor} (kt);
          \draw[human,->] (lt) -- node[right]{\scriptsize refactor} (kt);
        \end{pgfonlayer}
      \end{tikzpicture}
    }
    \endgroup
  \end{center}

  \caption[An overview of the MiTM ontology]{
    An overview of the MiTM ontology and how it interplays with the different systems. 
  }
  \label{fig:mitmontology}
  \end{figure}


We give a short overview of the structure of the Math-In-The-Middle ontology and how it is created. 
An overview can be seen in Figure~\ref{fig:mitmontology}. 

It illustrates the Math-In-The-Middle ontology and its interaction with three of the involved systems -- Sage, \lmfdb\ and GAP. 
Furthermore, we only show the concept of elliptic curves (or EC for short). 
This has been chosen because it can be found in all three systems in a well-developed form. 

Recall that everything is represented as a theory graph -- each rounded box inside the image represents a theory. 
Includes are shown as solid edges, meta-theory relations as dotted edges and views as wavy edges. 
Everything else represents explanation elements and are not part of the theory graph itself. 

The ontology is split into three parts. 
\begin{description}
  \item[Foundations]
    As the title suggests, this part of the MiTM ontology forms the basis for the others. 
    In the figure it is colored in green. 
    This part of the ontology defines the basic data types and knowledge types needed to formalize the other parts. 
    This in itself it split up in two parts, the mathematical foundations, and the system / programming language specific computational specific foundations. 

    On one hand, the mathematical foundations provide the basic mathematical objects needed -- logics and booleans, as well as arithmetics and (real) numbers. 
    This also includes a basic representation of Quantity Expressions such as $5 \frac{\mathrm{m}}{\mathrm{s}}$. 

    On the other hand computational foundations define the programming language equivalent of these, such as boolean values and IEEE floats. 
    On top of these, we define foundations for each of the systems. 
    This is again a layered approach, we start by defining the foundations of the given programming languages, e.g. in the case of python defining how inheritance works. 
    We then continue by formalizing the system foundation on top of this. 
    For example, we define how filters\footnote{In rough terms, filters are the GAP equivalent of classes. } in GAP function. 

  \item[Mathematics]
    This part of the MiTM ontology is the largest and colored in blue. 
    It contains the mathematical representation of objects that used by the systems. 
    This is the core idea behind Math-In-The-Middle, and as such they are not based on any specific system foundation, but instead use only the system independent mathematical and computational foundations. 
    
    In the figure, it can be seen that the mathematical concept of elliptic curves is defined. 
    Although not explicitly shown, this makes use of other concepts within the Math of MiTM. 

  \item[System Interfaces]
    This part of the ontology is the one most related to the systems themselves. 
    For each of the systems involved, a separate set of theories exists. 
    Each of these are shown in a separate purple cloud. 

    Consider for example the Sage System Interface (in purple on the left). 
    This defines a Sage specific representation of elliptic curves along with a view from the mathematical representation to it
    There is actually a partial view back from \identifier{Sage EC} back to the mathematical \identifier{EC}. 
    This is not shown here, but together with the depicted view this enables bi-directional translation of objects in Sage and objects
    in the Math part of the ontology. 
    This representation has the Sage Foundation as a meta theory, and can thus make use of Sage specific concepts. 
    
    A similar situation is the case for the GAP system, here of course we make use of the GAP meta-theory instead. 
    This is different for \lmfdb\ however. 
    Notice that there is no existing foundation for it. 
    This is because unlike the other systems, \lmfdb\ consists only a database and thus never performs any kind of computation and thus no computational foundation is needed. 
    Nonetheless, an interface theory for \lmfdb\ is needed. 
    
\end{description}

But how is this ontology created and maintained?
This is key to solving aspect~\ref{intro:rq:vt} of our research question. 
Consider the systems shown in the figure. 
With the exception of \lmfdb\ each of the systems contains a representation of the knowledge along with a set of (computational) algorithms of that knowledge and some form of database of objects\footnote{\lmfdb\ is a database with a MongoDB backend and thus does not contain a computational component. } . 

Whenever the representation of this knowledge is changed, it is necessary to update the system interface ontology. 
This is represented by the red arrows in the figure. 
Furthermore, the central mathematical representation needs to be able to express objects of all three involved systems. 
Hence to create this representation, one refactors the three representations into a single central one. 
This is again expressed by red arrows. 

\subsection{Overview of the SCSCP Protocol}\label{sec:mitm:scscp}

SCSCP \cite{SCSCP} stands for Symbolic Computation Software Composability Protocol. 
It is a protocol that enables communication between different mathematical systems and is based on a remote procedure call model. 
Clients can send calls to servers and each server has a fixed set of methods that are exposed to clients. 

In the context of our research question and the Math-In-The-Middle approach, the SCSCP protocol is used to facility communication between systems -- addressing aspect~\ref{intro:rq:com}. 
SCSCP is an Open Standard that and has been adopted by the OpenMath Society. 

All calls and values are represented as OpenMath objects. 
This makes it ideal for the MiTM approach -- recall that our formalization makes use of theory graphs which in this case already represent objects as OpenMath objects. 
Client and servers communicate by exchanging a combination of XML processing instructions and XML-encoded OpenMath objects via sockets. 

% !TEX root = ../thesis.tex
\begin{figure}[h]
  \begin{center}
    \begin{lstlisting}[language=XML]
<OMOBJ>
  <OMATTR>
    
    <!-- First a set of options for the function call -->
    <OMATP>
      
      <!-- some per-session unique idenfier -->
      <OMS cd="scscp1" name="call_id" />
      <OMSTR>fa99c557370a87b83a4ac16e10c43940</OMSTR>
      
      <!-- Tell the server that it should return the result of the computation as an object -->
      <OMS cd="scscp1" name="option_return_object" />
      <OMSTR></OMSTR>
    </OMATP>

    <OMA>
      
      <!-- indicate that a procedure call is being made -->
      <OMS cd="scscp1" name="procedure_call" />
      
      <OMA>
        
        <!-- the procedure to call --> 
        <OMS cd="scscp_transient_1" name="Addition"/>
        
        <!-- and the arguments to give to it -->
        <OMI>1</OMI>
        <OMI>1</OMI>
      </OMA>
    </OMA>
  </OMATTR>
</OMOBJ>
\end{lstlisting}
  \end{center}
  
  \vspace*{-1.5em}
  

  \caption[SCSCP Client Example]{
    An (annotated) OpenMath representation of an SCSCP function call to compute and return the result of $1 + 1$. 
  }
  \label{fig:scscpcall}
\end{figure}
Each call to the SCSCP Server includes a reference to the function symbol being applied and a set of parameters for it. 
Furthermore, every call can include a set of options that specify how the server should treat the result. 
It can either be sent back to the client on completion, stored on the server for reference in future calls, or discarded entirely. 
An example of an SCSCP call can be seen in Figure~\ref{fig:scscpcall} -- it shows and XML representation of the OpenMath SCSCP call. 
Details are not shown here -- instead the interested reader is referred to \cite{ODK-D3.3} for complete technical details.  

\input{figs/scscpclient.tex}
As \mmt\ implements theory graphs and thus OpenMath objects\footnote{
  As mentioned, MMT objects are OpenMath objects plus some custom extensions. 
  This means that some MMT objects are not strictly OpenMath objects and vice-versa. 
  For this reason, a standards-compliant OpenMath object layer has been implemented. 
  This enables efficient translation between MMT terms and OpenMath objects within Scala. 
}, it is ideal for implementing the SCSCP protocol. 

During the progress of this thesis, an SCSCP Server and Client have been implemented in \mmt, both written in Scala. 
The client functionality allows \mmt\ to first encode an \mmt\ term as an OpenMath object, then have any SCSCP server perform a computation on this object, and finally receive the response. 
This response can then be translated back into an \mmt\ term. \
An example of the client functionality can be found in Figure~\ref{fig:scscpclient}
The server functionality allows \mmt\ to expose computational functionality to other systems.
Again, the server first receives an OpenMath object, translates it into an \mmt\ term, performs some computation, and then returns the result as an OpenMath object. 