% !TEX root = ../thesis.tex
\appendix
\section{Detailed Contributions And Acknowledgements}\label{sec:contrib}

I give a short overview of which parts of the work described in this thesis have been created only by me, and which have been created collaboratively together with others. 
This has already been acknowledged with citations. 

\paragraph*{Section~\ref{sec:intro}}
The preliminaries, like the \omdocmmt\ infrastructure, the OpenDreamKit project, and the Math-In-The-Middle approach, have each been worked on collaboratively with others. 
In particular, Figure~\ref{fig:classicalconnect}, Figure~\ref{fig:mitmconnect}, as well as the example presented in Figure~\ref{fig:mitmcaseintro} have been adapted from different OpenDreamKit project papers, written collaboratively with other project members. 
The research question, as well as the extensions to the QMT Query Language, have been contributed by myself for the use within this thesis. 

\paragraph*{Section~\ref{sec:mmt}}
In this section I detail backgroud information about the \mmt\ system, especially for readers not familiar with it. 
The \mmt\ system, and the \omdocmmt\ infrastructure, as indicated, was developed within the KWARC group, however this section has been written from the ground up by me. 
It re-uses a frequent example from within the research group; Figure~\ref{fig:theorygraph} is an adapted version of a previously existing Figure. 

\paragraph*{Section~\ref{sec:mitm}}
In this section, I related the Math-In-The-Middle approach to the research question. 
The Math-In-The-Middle approach itself was not invented by me, it been designed together with other members of the OpenDreamKit project. 
Figure~\ref{fig:mitmontology} is a customized version of a Figure that has appeared in different OpenDreamKit project presentations. 
The SCSCP protocol existed prior to the OpenDreamKit project, and has been originally designed by others. 
Within the scope of this thesis, I have implemented the protocol within \mmt\ along with a pure OpenMath implementation on my own.  

\paragraph*{Section~\ref{sec:vt}}
In this section, I describe the structure of \lmfdb\ and describe the implementation of Virtual Theories. 
\lmfdb\ is a database that is maintained by others, and as such I do not claim to have developed the structure, or found the knowledge contained within it. 
The codec and virtual theory infrastructure described here has again been developed collaboratively in the KWARC group; some parts of the implementation have been written by others. 
All Figures within this Section are my own. 

\paragraph*{Section~\ref{sec:qmt}}
In this section, I describe the QMT Query Language, and a few syntax extensions. 
The language itself, and the basic implementation for concrete theories, has been developed by the KWARC group, and in particular Florian Rabe. 

\paragraph*{Section~\ref{sec:comm}}
In this section I describe my extensions to the QMT Query Language. 
These include evaluation within Virtual Theories, the syntax on the MMT surface level, and the web interface. 
I have developed and implemented these, and as such all Figures and Screenshots in this Section are my own. 

\paragraph*{Section~\ref{sec:conclusion}}
In this section I revisit the collaboratively developed use-case, and then give an outlook and a conclusion. 
Figure~\ref{fig:mitmcase} is again heavily adapted from an OpenDreamKit presentation, all other text and Figures are my own. 