% !TEX root = ../thesis.tex
\section{The QMT Query Language}\label{sec:qmt}

Now that we have seen how to represent and maintain knowledge in the MiTM ontology, we can turn towards aspect~\ref{intro:rq:ql} of the research question.  
There is a multitude of problems where is is not enough to give the URI of a specific declaration and then retrieve this via \mmt. 
There are several queries where one either does not know the exact URIs of all items to be retrieved, or where one wants to find all declarations subject to a specific criterion. 

To allow for more flexibility and enable general ``information retrieval'' \mmt\ has a query language.
This query language is called QMT -- Query language for Mathematical Theories.  
QMT started as a query language for formal libraries \cite{Rabe:qlfml12}, but within the scope of this project it has been extended to better fit our needs.

The query language enables the user to find a set of objects -- this can be \mmt\ objects (i.e. terms), declarations or even a set of theories and views -- subject to specific criteria.
\begin{example}\label{example:simple}
	One simple use case is to find all declarations inside of a given theory -- for example the query
	\begin{lstlisting}[language=qmt]
Related(
	Literal(URI('odk:algebra?Semigroup')),
	ToObject(Declares())
)
	\end{lstlisting}
	when evaluated returns the three URIs to $G, \circ$ and \identifier{assoc} from Figure \ref{fig:theory}.
\end{example}
Understanding even this simple query is non-trivial. 
We leave it to the below once we have gone over some parts of the query language to reiterate this example.

\subsection{The Inductive Nature of the Query Language}\label{sec:qmt:inductive}

The first important thing to note is that the query language is an inductive data structure.
This decision was made for two reasons, (1) to enable typing of the language and (2) to allow easier implementation of evaluation.
Type-checking of the language can prove very useful in practice, for instance to check if one has formulated a query correctly. 
It is also used during evaluation -- but more on this later.

Being inductive in nature, the query language contains certain basic queries, that take few to no parameters, and complex queries, that take queries themselves as parameters. 
The query language is able to express information about different forms of content inside of \mmt. 
It can return URIs, Objects and String literals -- we refer to these as \identifier{QueryBaseType}s.
Example~\ref{example:simple} above is asking for a set of URIs -- namely all the URIs that are declared within the theory \uri{odk:algebra?Semigroup}.

Different queries may return different types of results.
Specifically, a query can return either a typed tuple of \identifier{QueryBaseType}s  or a homogeneous set of such tuples
\footnote{In this setting ``typed tuple'' refers to the exact typing of each element in the tuple.
For example, we consider a tuple of type \inlinecode{(URI, Object)} as typed but not one of type \inlinecode{(QueryBaseType, QueryBaseType)}. }.

\begin{example}\label{example:divisible}
	Another question one might ask is, ``what are the divisors of elliptic curves inside of \lmfdb\ that have a conductor divisible by 5''.
	In fact, this question came up during the OpenDreamKit project when asking John Cremona about what queries the existing \lmfdb\ architecture could not handle\footnote{
		It turns out that this query could actually be handled by existing architecture.
		However, this was not an intended feature and a side-effect of an arbitrary code execution vulnerability on the side of the server.
	}.
\end{example}

Such a query can be expressed inside the query language -- although it gets very complex very quickly -- thus we omit it here. 
Unlike the previous example, this one returns a set of \mmt\ objects -- namely those representing divisors.

The grammar of the query language consists of several parts: \identifier{Queries}, \identifier{Propositions}, \identifier{Relations}, \identifier{Predicates} and \identifier{Judgements}.
An overview of each of these is given in the following, in particular detailing which queries these can be used to answer.

Inside the query language, the \identifier{Query} class represents evaluable queries.
The structure of queries is very much related to evaluation and intentionally kept restricted in certain situations.
They can be roughly classified into three different types:
\begin{enumerate}
	\item Terminal Queries, which do not have sub-queries as parameters, 
	\item Complex queries, formed by query operators which that take a sub-query as a parameter and upon evaluation evaluate the subquery and use the result thereof, and
	\item Technical Queries, which have no direct influence on the result but only on how the query is evaluated
\end{enumerate}

The aim of this section is to give the reader an overall understanding of the query language, thus some cases are omitted. 
The interested reader is referred to the \mmt\ website \cite{uniformal:on} and in particular the source code for a complete list. 
Furthermore, the following talks about evaluating queries and at the same time treats them as functions that have a return value -- both of these refer to the same thing.

\subsubsection*{Terminal Queries and Unary Predicates}\label{sec:qmt:terminal}

The first type of terminal queries are those that return object literals -- appropriately called \inlinecode{Literal($\meta{obj}$)} and \inlinecode{Literals($\meta{objs}$)}.
For example, a query \inlinecode{Literal(URI('odk:algebra?Semigroup'))} directly evaluates to the URI literal \inlinecode{URI('odk:algebra?Semigroup')}.

Unary predicates effectively function as types of \mmt\ URIs -- they state which kind of objects the URIs refer to.
For example, the unary predicate \inlinecode{IsTheory()} applies to URIs like \uri{odk:algebra?Semigroup} and \uri{odk:algebra?Algebra} -- all URIs that contain exactly one ``?'' and thus point to theories.
Other important unary predicates include include \inlinecode{IsView()} and \inlinecode{IsConstant()}.

Unary predicates form the basis of the second type of terminal queries -- the \inlinecode{Paths($\meta{un}$)} query.
\begin{example}\label{example:paths}
	The query \inlinecode{Paths(IsTheory())} returns the paths to all theories known to \mmt.
\end{example}
In general, path queries return the set of paths that are subject to the given unary predicate.

As queries are inductive in nature, these two terminal queries form the basis of all queries.
This choice was made intentionally -- it enables a functional implementation of the query language to be contained in a certain sense.
A functional implementation has to evaluate each sub-query individually and at one point have each of the intermediate results in memory. 
This choice makes sure that the sets each implementation has to work with are limited -- one never has to work with all paths in \mmt\ at once.

\subsection{Query Results as Sets}\label{sec:qmt:results}

The second type of queries are \textit{Complex Queries} made up of \textit{Query Operators}. 
These take queries as parameters and return a new query.

\begin{example}\label{example:unions}
The query \begin{lstlisting}[language=qmt]
Union(
  Paths(IsTheory()), 
  Paths(IsView())
)
  \end{lstlisting} returns the set of all paths that are either theories or views.
\end{example}
This example demonstrates an operator that treats query results as sets -- simply lifting the union operator to the query level.
These types of queries do not depend on the semantics of content in \mmt.
Apart from the \inlinecode{Union($\meta{left}$, $\meta{right}$)} operator, complex queries can also be formed by \inlinecode{Intersection($\meta{left}$, $\meta{right}$)} and \inlinecode{Difference($\meta{left}$, $\meta{right}$}.

\subsubsection*{Making Use of Binary Predicates}\label{sec:qmt:binary}

On top of unary predicates of \mmt\ paths there are also binary predicates.
As binary predicates do these relate two \mmt\ URIs to each other -- for example the binary predicate \identifier{Declares} relates the subject \uri{odk:algebra?Semigroup} to the object \uri{odk:algebra?Semigroup?G}.
Other binary predicates include \identifier{Declares}, \identifier{HasDomain}, \identifier{HasCodomain} and \identifier{Includes}.

To be able to use these predicates inside of queries the \inlinecode{Related($\meta{to}$, $\meta{by}$)} operator exists.
We have already seen an example for this, Example~\ref{example:simple} above was the first example introduced.
This demonstrates the use of this operator -- the $\meta{to}$ parameter generates a set of URIs and is then filtered by everything related by a given relation $\meta{by}$.

It should be noted that a single binary predicate can not directly function as a relation.
To elaborate this further, consider a query that returns the single URI \uri{odk:algebra?Semigroup} and set it as the $\meta{to}$ parameter for a \inlinecode{Related} query.
Looking into the \identifier{Declares} relation one has two choices -- find all URIs which are declared by this theory or find all URIs which declare this theory.
\identifier{ToSubject} and \identifier{ToObject} can be used to clarify which one is requested. 

\subsubsection*{Binding Variables in Queries}\label{sec:qmt:binding}

Another type of query operators allows for binding of variables. 
These created named references to intermediate results of the query and have undergone a refactoring and expansion during this project. 
The most relevant query operator is the \inlinecode{Comprehension($\meta{domain}$, $\meta{varname}$, $\meta{pred}$)} -- which allows to filter a specific domain by those that fulfill a specific predicate.


\begin{example}\label{example:filter}
	The query \begin{lstlisting}[language=qmt]
Comprehension(
	Related(
		URI('odk:algebra?Semigroup'), 
		ToObject(Declares())
	), 
	LocalName('x'), 
	Holds(
		Bound(x), 
		Typing('x', OMV(LocalName('x')), OMID(odk:algebra?Semigroup?G))
	)
)
	\end{lstlisting}
	finds all declarations within the \textit{Semigroup} theory that have a type 'G'.
\end{example}

We start on the left hand side by finding all URIs to declarations within the \textit{Semigroup} theory. 
This is the same as in Example~\ref{example:simple}.
Next, we use the comprehension operator to find all of those $x$s which are subject to a specific predicate.
In general, \inlinecode{pred} parameter refers to a predicate, which in the codebase is represented by a class on its own.

Here we specifically make use of the \inlinecode{Holds($\meta{about}$, $\meta{j}$)} predicate, which has been added to the grammar language to allow interaction with the object level of \mmt. 
It can check if a so-called \identifier{Judgement} holds for the current variable ($x$).
We declare a local variable within the query called $x$ inside the Comprehension operator. 
We then refer to it within the predicate by using using \inlinecode{Bound(x)}.
If we did not allow for binding of variables, we would only be able to declare predicates that are universally true or false -- resulting in the Comprehension operator becoming useless. 

Other kinds of queries that allow binding include \begin{itemize}
	\item the \inlinecode{Let($\meta{varname}$, $\meta{value}$, $\meta{in}$)}, which explicitly make a variable available,  
	\item the \inlinecode{BigUnion($\meta{domain}$, $\meta{varname}$, $\meta{of}$)} operator, and
  \item the \inlinecode{Map($\meta{varname}$, $\meta{in}$, $\meta{term}$)}, which constructs a new term for each element of the original query. 
\end{itemize}
The last \identifier{Map} query has been added during this thesis to make certain computational tasks easier to express. 

Judgments, as given to the predicate by the $j$ parameter, are represented by the \identifier{QueryJudgment} class.
This is a new addition to QMT. 
It allows users to dynamically look into the \mmt\ types and structure of objects and check if they are subject to certain conditions. 
In this specific example we use the \identifier{Typing} Judgment to check if a variable has a type $G$.

\subsection{Evaluating Queries of MMT Content}\label{sec:qmt:eval}

Next, the reader is provided with an overview of how queries are evaluated. 
This exploits the typing of the query language, in particular return value of the query can be predicted without having to evaluate the query.

Recall that \mmt\ queries can return either a typed tuple of \inlinecode{QueryBaseType}s or a list thereof.
This allows to have a query always return a set of tuples of \inlinecode{QueryBaseType} internally. 
A single tuple \inlinecode{QueryBaseType} can be represented by an appropriate singleton -- while separately maintaining the right typing for the results.

We use the inductive nature of queries to guide us in the evaluation.
Here a single query is considered as a (syntactic) tree, and the leaves are used start an inductive evaluation procedure. 
During this procedure, a big initial result set is generated for the terminal queries, and then used inductively to make this set smaller.
As mentioned above the terminal queries are intentionally limited -- to allow us to keep all of these intermediate results in memory.

Recall that two important terminal queries are the \identifier{Literal} and \identifier{Paths} queries.
The first query of these is trivial to evaluate -- simply place the literal inside of the query into the intermediate result set -- whereas the second one is a bit more tricky.

\subsubsection*{Evaluation of Predicates and Relations}\label{sec:qmt:predeval}

To evaluate a \identifier{Paths} query -- like the one in Example~\ref{example:paths} -- all URIs that are subject to a given unary predicate need to be found
Assuming that all content is stored on disk, this can be solved in an elegant fashion.

Whenever new content is added to \mmt, it is indexed and all information about which objects are of which type are added to an index. 
This index comes in the form of a \identifier{RelationalStore}, which is managed by a \identifier{RelationalManager} (the naming will become more obvious later).
At evaluation time, the implementation can easily look into this index and quickly retrieve all URIs of a specific type. 

Next, the evaluation of query operators is explained. 
Only two of four cases are non-trival. 

In the case where these operators simply lift set operations to operations on queries, the evaluation is straightforward -- the set operations are just performed on the intermediate results.
Now for the interesting cases, we have a look at the \inlinecode{Related($\meta{to}$, $\meta{by}$)} operator.
It has to be aware of binary predicates.
The implementation makes use of the same trick as above with the unary predicates -- whenever new content is added, all possible binary predicates are cached and store in an index. 
This again happens inside of the \inlinecode{RelationalStore}. 
Now the name makes sense as it stores relational information about a given set of knowledge items.
During evaluation, the index can be used to quickly retrieve all items related to a given URI with a given predicate.

Thus the query in Example~\ref{example:simple} is evaluated in two steps. 
First, one evaluates the argument \inlinecode{Literal(URI('odk:algebra?Semigroup'))} -- it is a literal query so it just returns the argument. 
Then one looks into the \inlinecode{RelationalStore} and finds all stored binary predicate triples, where the relation is \inlinecode{Declares} and the object is the URI \uri{odk:algebra?Semigroup}. 
This will give exactly what we the query is looking for -- the URIs to constants declared by the \textit{Semigroup} theory from Figure~\ref{fig:theory}.

\subsection{Evaluation of Predicates Involving Bound Variables}\label{sec:qmt:boundeval}

Next, the evaluation of the \inlinecode{Comprehension($\meta{domain}$, $\meta{varname}$, $\meta{pred}$)} operator is described. 
For this, recall Example~\ref{example:filter}. 
Like with the other query operator the argument needs to be evaluated first. 
This is the exact query given by Example~\ref{example:simple} and returns a set of URIs to concepts. 

Next, an iteration is required. 
For each element of this intermediate result set the other important argument to the query -- a predicate -- needs to be evaluated. 
Specifically, it needs to be checked if the \begin{lstlisting}[language=qmt]
Holds(
	Bound(x), 
	Typing('x', OMV(LocalName('x')), OMID(odk:algebra?Semigroup?G))
)
\end{lstlisting} predicate applies or not. 

For this purpose a sub-query is spawned multiple times -- one for each of the elements inside the original set. 
These sub-queries evaluate the predicate and need to know which element of the intermediate results the $x$ stands for. 
Thus it needs to be given a context. 
In this case, a context is simply a set of mappings for variables. 

The QueryJudgement \begin{lstlisting}[language=qmt]
Typing('x', OMV(LocalName('x')), OMID(odk:algebra?Semigroup?G))
\end{lstlisting}
needs to be evaluated. 
Principally, this can be achieved by a part of \mmt\ called the Solver. 

The actual evaluation is rather technical and first a so-called \identifier{Stack} for the Judgment is needed. 
The Stack contains a set of rules that can be used to evaluate the judgment. 
To hide the process of having to retrieve this Stack first from the User, the QueryJudgment class has been introduced. 
This automatically retrieves all needed information at query evaluation time and passes on the information if the judgment holds or not from the Solver. 

After the evaluation of sub-queries has been finished, is is easy to generate the result set. 
Only the set of elements for which the predicate applies -- which are known at this point -- should be returned. 
Other queries that bind variables make use of a similar procedure, with the exception that they do not check if a predicate applies, but instead evaluate different subquery results.