% !TEX root = ../thesis.tex
\section{Overview of the MMT System}\label{sec:mmt}

\mmt\ \cite{Rabe:MMTLanguageSystem09} is a language and framework that was developed to manage (mainly) mathematical knowledge and can handle both formal and informal content.
It is foundation independent, i.e. it does not enforce a certain logic for formal content.

It consists of two main parts, the \omdocmmt\ language and the \mmt\ System.
The \omdocmmt\ language \cite{rabe:mmt:09} forms the basis of knowledge representation.
The \mmt\ System \cite{Rabe:MMT}, sometimes also referred to as the \mmt\ API, is the implementation of this language written in Scala.
It can be extended by custom plugins.
The project is under active development and the source code can be retrieved via \cite{uniformal:on}.

\subsection{Theories as Sets of Declarations}\label{sec:mmt:theories}

Knowledge in \omdocmmt\ is organized in theories which group knowledge together semantically.
They consist of a set of declarations, each of which has a name as well as optionally a type, a definition and other kinds of meta data.
A simple theory about Semigroups can be found in Figure~\ref{fig:theory}.

% !TEX root = ../thesis.tex
\begin{figure}[h]
  \begin{center}
    \begin{tabular}{|l c l|}
      \hline
      \textsf{Semigroup} $:$ \textsf{LF} & &\\\hline
      $G$ & $:$ & $ \typett$\\
      $\circ$ & $:$ & $ G \rightarrow G \rightarrow G$\\
      $ \mathsf{assoc}$& $:$ & $ \plaintt{ded}\left( \forall x \in G . \forall y \in G . \forall z \in G . (x\circ y)\circ z=x\circ (y\circ z) \right)$\\\hline
    \end{tabular}
  \end{center}

  \caption[A sample Theory of Semigroups]{
    A Theory of Semigroups.
  }
  \label{fig:theory}
\end{figure}

We start by giving the theory a name -- Semigroup in this case -- and declaring the logical foundation we are using to formalize it -- in this case the LF Logical Framework \cite{Pfenning91}.
The logical foundation used is called a meta-theory and is in itself an \omdocmmt\ theory\footnote{
    In contrast to normal theories, meta-theories like LF within the \mmt\ system are commonly implemented in the form of code, rather than just taking the form of declarations.
    This allows implementation of type checking as well as other \mmt\ features.
}.

In \mmt, each theory resides within a specific namespace -- which is not explicitly shown here.
The namespace -- for example \url{http://www.opendreamkit.org/algebra} -- together with the theory name \inlinecode{Semigroup} forms the \mmt\ URI of the theory \url{http://www.opendreamkit.org/algebra?Semigroup}.
This URI can be used to refer to the theory.

To prevent having to write down long urls when they are used frequently \mmt\ supports the concept of URI abbreviations. 
These function in a straightforward fashion. 
For example, instead of constantly having to write \url{http://www.opendreamkit.org/} one define an abbreviation \uri{odk:} and then reduce URIs like \url{http://www.opendreamkit.org/algebra?Semigroup} to \uri{odk:algebra?Semigroup}. 

% !TEX root = ../thesis.tex
\begin{wraptable}{r}{0.5\textwidth}\
  \begin{tabular}{|r|l|}
    \hline
    \textsf{Abbreviation} & \textsf{URI} \\\hline\hline
    \identifier{meta:} & \url{http://cds.omdoc.org/meta} \\
    \identifier{qmt:} & \url{http://cds.omdoc.org/qmt} \\
    \identifier{odk:} & \url{http://www.opendreamkit.org/} \\
    \identifier{mitm:} & \url{http://mathhub.info/MitM/} \\
    \identifier{lmfdb:} & \url{http://www.lmfdb.org/} \\\hline
  \end{tabular}

  \caption[List of URI abbreviations]{
    A list of URI abbreviations used in \mmt. 
  }
  \label{fig:abbrevs}
\end{wraptable}
A list of common namespace abbreviations can be found in Table~\ref{fig:abbrevs}. 
From this point forward, any URI makes use of these abbreviations. 

We continue by making three declarations.
These represent the requirements for semigroups: A set of elements, a (closed) operation on them and the axiom that the operation is associative.
$G$, which is declared as a type, represents the set of elements.
Declaring it as a type allows us to use it in the declarations below.

Note that, like above, the name together with the URI for the theory provides a URI to the content being defined.
Here this is \uri{odk:algebra?Semigroup?G}.
\mmt\ URIs to declarations in general consist out of a triple $(d, m, s)$. 
$d$ is a path to a namespace\footnote{
  $d$ as in document.
  These are essentially namespaces -- but like theories form groups of semantically related declarations so do documents form semantic groupings for theories.
} (here \uri{odk:algebra}), 
$m$ is the name of a theory\footnote{
  The $m$ here stands for module -- a theory or a view.
  We will define what a view is later on.
} (here \inlinecode{Semigroup}) , 
and $s$ is the name of a symbol (here $G$)

Most of the time \mmt\ URIs are straightforward -- however in certain situations are not.
Principally \mmt\ allows arbitrary nesting of theories -- that is theories within theories.

Next, $\circ$ is declared and assigned the type $G \rightarrow G \rightarrow G$.
Informally, this declares $\circ$ as a function taking two elements of $G$ and returning another element of $G$. 
This makes it a closed operation on the semigroup elements.
Finally, we declare the axiom of associativity called \inlinecode{assoc}.
Here we make use of the \textit{ded} symbol via the Curry-Howard isomorphism\cite{Howard:tfnoc80}.
Intuitively, the symbol returns the type of all proofs for the statement in question.
Since we are declaring a constant of this type, this means that there is a proof for associativity -- making it an axiom.

Types and definitions of objects make up the object layer.
\mmt\ objects, also called terms, are essentially OpenMath \cite{BusCapCar:oms04} objects with some custom extensions.
As such, they can consist of 
\begin{itemize}
    \item references to symbols (for example, we just use $\circ$ above, but a formal system needs to know what $\circ$ is), 
    \item literals (mostly Integers like $1, 2, \dots$, but also strings), 
    \item variables and constants, 
    \item applications of terms to other terms ($G \rightarrow G$ is really the symbol $\rightarrow$ applied to two arguments $G$ and $G$), and
    \item bindings of variables inside of terms (for example in terms like $\forall X. P(X)$, the variable $X$ is bound inside of the $\forall$ term). 
\end{itemize}

\mmt\ terms can be parsed both from an XML syntax as well as from a human-readable and writable surface syntax. 
The surface syntax can be further defined by so-called notations. 
These are additional meta-data on declarations that specify how a specific symbol or application of a symbol should look like. 

\subsection{Structure of Theory Graphs}\label{sec:mmt:tg}

Theories can be related via three different relations, imports, structures, and views.
Imports make knowledge, i.e. all declarations, from one theory available in another and are used to extend theories.
Views are truth-preserving mappings from declarations in one theory to another.
Structures are very similar to views, except that they allow for renaming of declarations. 
Together Theories, Imports and Views naturally form a graph-like structure call a Theory Graph.
An example of a theory graph -- extending the example above -- can be found in Figure~\ref{fig:theorygraph}.

% !TEX root = ../thesis.tex
\begin{figure}[h]
  \begin{center}
    \begin{tikzpicture}[xscale=3.9,yscale=2.2]\footnotesize
      \node[thy] (sg) at (1,-1) {
        \begin{tabular}{l}
          \textsf{Semigroup}\\\hline
          $G,\circ$\\\hline
          $\scriptstyle x\circ y\in G$ \\\hline
          $\scriptstyle \plaintt{assoc}: (x\circ y)\circ z=x\circ (y\circ z)$
        \end{tabular}
      };
      \node[thy] (m) at (1,0) {
        \begin{tabular}{l}
          \textsf{Monoid}\\\hline
          $e$\\\hline
          $\scriptstyle e\circ x=x$
        \end{tabular}
      };
      \node[thy] (g) at (1,1) {
        \begin{tabular}{l}
          \textsf{Group}\\\hline
          $i := \scriptstyle\lambda x.\tau y. x\circ y=e$\\\hline
          $\scriptstyle\forall x:G.\exists y:G.x\circ y=e$
        \end{tabular}
      };
      \node[thy] (cg) at (1,2.3) {
        \begin{tabular}{l}
          \textsf{Commutative Group}\\\hline
          \\\hline
          $\scriptstyle \plaintt{comm} : x\circ y=y\circ x$
        \end{tabular}
      };

      \node[thy] (N) at (-1,-1) {
        \begin{tabular}{l}
          \textsf{Natural Numbers}\\\hline
          $\mathbb{N},s,0$\\\hline
          $\scriptstyle P1$,\ldots $\scriptstyle P5$
        \end{tabular}
      };
      \node[thy] (Np) at (-1,0) {
        \begin{tabular}{l}
          \textsf{Natural Numbers with Plus}\\\hline
          $+$\\\hline
          $\scriptstyle n+0=n$,\\
          $\scriptstyle n+s(m)=s(n+m)$
        \end{tabular}
      };
      \node[thy] (Nt) at (-1,1) {
        \begin{tabular}{l}
          \textsf{Natural Numbers with Multiplication}\\\hline
          $\cdot$\\\hline
          $\scriptstyle n\cdot1=n$,\\
          $\scriptstyle n\cdot s(m)=n\cdot m+n$
        \end{tabular}
      };
      \node[thy] (ia) at (-1,2.3) {

        \begin{tabular}{l}
          \textsf{Integer Arithmetics}\\\hline
          $-$\\
          $\mathbb{Z} := \mathbb{N}\cup-\mathbb{N}$\\\hline
          $\scriptstyle-0=0$
        \end{tabular}
      };

      %Right side
      \draw[include] (sg) -- (m);
      \draw[include] (m) -- node[left] (mg) {$\mathsf{g}$} (g);
      \draw[include] (g) -- (cg);

      %Left side
      \draw[include] (N) -- (Np);
      \draw[include] (Np) -- (Nt);
      
      \draw[struct] (Nt) to[bend left=10] node[left] {$p$} (ia);
      \draw[struct] (Nt) to[bend right=10] node[right] {$n$} (ia);  

      %Links
      \draw[view] (m) -- node[above] {$
      \thmo{f}{
      \left\{\begin{array}{l}
        G\mapsto\mathbb{N}\\
        \circ\mapsto +\\
        e\mapsto 0
      \end{array}\right\}
      }$
      } (Np);
      \draw[view] (cg) -- node[above] {$\thmo{h}{
      \left\{\begin{array}{l}
        i\mapsto -\\
        \mathsf{g}\mapsto\mathsf{f}
      \end{array}\right\}
      }$} (ia);
    \end{tikzpicture}
  \end{center}

  \caption[A Simple Theory Graph]{
  Simplified example of a theory graph. 
  Imports and Structures are represented as solid edges and views as wavy edges. 
  The mappings of the views are given explicitly. 
  We omit the complete definitions of the Peano axioms for the natural numbers.
  }
  \label{fig:theorygraph}
\end{figure}

We start on the bottom right by declaring a semigroup as above.
Due to complexity of the theory graph, we use more simplified notation and omit some details like meta-theories.
In Mathematics, a Monoid can be defined as a semigroup together with an identity element.
This is exactly what happens here as well, using an Import to from the semigroup theory into the Monoid theory and then adding declarations for the identity element $e$.
We continue with this principle, first we extend the Monoid into a Group by adding an identity element $i$ together with an appropriate axiom and finally we extend this into a commutative group (top right) by adding an axiom of commutativity.

We also make use of the other relation between theories, a so-called view.
This is done on the left hand side of the figure.
Here we start on the bottom by defining the natural numbers, as formalized by the Peano Axiomes\footnote{
For simplicity, we omit their exact description here.
They are merely referred to as $p_1, p_2, p_3, p_4, p_5$. }.
The next important step is the addition of a plus operation, again through the principle of importing the existing theory and adding another declaration.

Any skilled mathematician will have noticed that Natural Numbers together with addition are an example of a Monoid, in other words one can ``view'' the natural numbers as a Monoid.
To express this formally, one has to be a bit more explicit.
The set of elements $G$ correspond to the natural numbers $\mathbb{N}$, the operation $\circ$ corresponds to $+$ and the identity element $e$ is given by $0$.
This mapping is given by the view $g$ and is truth-preserving in a certain sense: Everything that is true based on the axioms inside the $Monoid$ theory still holds in our Natural Numbers example -- after translation via the view of course.

On the top left hand corner we make use of this mechanism even more. 
We would first like to make use of the Imports mechanism to create a theory of ``Integer Arithmetics''. 
But there is a problem: We need the natural numbers twice, one for positive numbers, once for negative numbers. 
To achieve this, we make use of structures, called $p$ and $n$ here. 
$p$ imports the natural numbers as positive integers, and $n$ imports them again as negative numbers. 
We then need to add the axiom that the zeros from both imports are actually the same. 

Furthermore, we use the view $h$ to show that this is an example of a ``Commutative Group''. 
The first line of the mapping is not too interesting -- we just state the inverse is given by the negation of an integer -- but something special happens in the second line.
Note how we have not yet given mappings for a lot of the elements, for example groups still require a base set $G$ which is formally known because of the Import.
Instead of giving this mapping again, we instead take the import of the group theory $g$ and map it via an existing view $f$.
Similar to how theories are modularized via Imports, this allows us to structure views in a similar fashion. 

\subsection{Accessing Content in MMT by URI}\label{sec:name}

The primary method for accessing knowledge in \mmt\ is by name.
Recall that all content, including theories, views and declarations have been assigned a URI.
These URIs function like a name -- each of them identifies a piece of content inside of \mmt.

Consider for example, an author wishing to retrieve the associativity axiom for the \textit{Semigroup} theory we defined in Example~\ref{fig:theorygraph}.
Assuming that the author is familiar with our naming scheme, they can identify the URI \uri{odk:algebra?Semigroup?assoc}.
They can then directly access this URI in \mmt.

Another example of accessing a knowledge item comes from the OpenDreamKit project.
During the course of the project, we have in particular made several curves stored within the \lmfdb\ database available to \mmt.
Consider the elliptic curves dataset.
Within \mmt, this has been given the URI \uri{lmfdb:db/elliptic_curve}.
Each curve has a label assigned to by \lmfdb\ itself, for example \inlinecode{11a1}.
Should a user want to retrieve information about this specific curve, they can us the URI \uri{lmfdb:db/elliptic_curve?11a1}.

\subsection{Using URIs for Knowledge Representation}\label{sec:name:knowledge}

Consider the \inlinecode{Monoid} theory from Figure~\ref{fig:theorygraph}.
Imagine we want to retrieve the associativity axiom for Monoids -- can we just ask for the URI \uri{odk:algebra?Monoid?assoc}?
The immediate answer is no -- there is no such declaration as it is only available via the \textit{Import}.
Thus the retrieval should fail.

In this example, one can easily retrieve the \uri{odk:algebra?Semigroup?assoc} instead.
We are familiar with the theory graph structure and can just follow the import.
In other situations, the theory graph structure might not be as well-known by the person wanting to retrieve a declaration -- or it might not be obvious.
This is the case for the \textit{Integer Arithmetics} theory -- here we are ``importing'' the same theory twice via two different structures.
Even if we want to look for the associativity axiom in this theory and made \textit{Imported declarations} available it would not be clear how to make a difference between the two different associativity axioms.

% !TEX root = ../thesis.tex
\begin{figure}[h]
  \begin{center}
    \begin{tabular}{|l c l|}
      \hline
      \textsf{MonoidFlat} $:$ \textsf{LF} & &\\\hline
      $G$ & $:$ & $ \typett$\\
      $\circ$ & $:$ & $ G \rightarrow G \rightarrow G$\\
      $\plaintt{assoc}$& $:$ & $ \plaintt{ded}\left( \forall x \in G . \forall y \in G . \forall z \in G . (x\circ y)\circ z=x\circ (y\circ z) \right)$\\
      $e$ & $:$ & $G$\\
      $\plaintt{ident}$& $:$ & $ \plaintt{ded}\left( \forall x \in G x\circ e = e\circ x = x \right) $\\
      \hline
    \end{tabular}
  \end{center}

  \caption[A flattened Theory of Monoids]{
    An elaborated, flattened version of the Monoid theory from Figure~\ref{fig:theorygraph}. 
  }
  \label{fig:flattheory}
\end{figure}


This is where the concept of flattening steps in.
Flattening formalizes what we have only explained in a hand-wavy fashion so far -- taking an imported (or structurally made available) theory and replacing it with explicit declarations.
The process of flattening is sometimes also referred to as elaboration and an example can be found in Figure~\ref{fig:flattheory}.
We can see that this theory no longer has an import of the \textit{Semigroup} theory -- instead we have the declarations $G, \circ$ and \textit{assoc}.
Such theories are never written explicitly but instead are built by the \mmt\ system internally -- thus allowing us to retrieve \uri{odk:algebra?Monoid?assoc} after all.

In summary, URIs are very useful -- both in flattened and non-flattened form -- when it comes to knowledge representation.
This is only the case if one knows the structure of the knowledge one is talking about, as they enable us retrieve a specific knowledge item.