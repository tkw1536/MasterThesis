% !TEX root = ../thesis.tex
\begin{figure}[h]
  \begin{center}
    \begin{lstlisting}[language=scala]
// import the SCSCPClient and OpenMath libraries
import info.kwarc.mmt.odk.SCSCP.Client.SCSCPClient
import info.kwarc.mmt.odk.OpenMath._

// establish a connection
val client = SCSCPClient("scscp.gap-system.org")

// get a list of supported symbols
/** List(OMSymbol(Size,scscp_transient_1,None,None),
 * OMSymbol(Length,scscp_transient_1,None,None),
 * OMSymbol(LatticeSubgroups,scscp_transient_1,None,None),
 * OMSymbol(NrConjugacyClasses,scscp_transient_1,None,None),
 * OMSymbol(AutomorphismGroup,scscp_transient_1,None,None),
 * OMSymbol(Multiplication,scscp_transient_1,None,None),
 * OMSymbol(Addition,scscp_transient_1,None,None),
 * OMSymbol(IdGroup,scscp_transient_1,None,None),
 * ...,
 * OMSymbol(NextUnknownGnu,scscp_transient_1,None,None)) */
println(client.getAllowedHeads)

// We make a simple example: Apply the identity function to an integer 1
val identitySymbol = OMSymbol("Identity", "scscp_transient_1", None, None)
val identityExpression = OMApplication(identitySymbol, List(OMInteger(1, None)), None, None)

/** OMInteger(1,None) */
println(client(identityExpression).fetch().get)

// We also try to compute 1 + 1
val additionSymbol = OMSymbol("Addition", "scscp_transient_1", None, None)
val additionExpression = OMApplication(additionSymbol, List(OMInteger(1, None), OMInteger(1, None)), None, None)

/** OMInteger(2,None) */
println(client(additionExpression).fetch().get)

// and close the connection
client.quit()
\end{lstlisting}
  \end{center}
  
  \vspace*{-1.5em}
  

  \caption[SCSCP Client Example]{
    This example shows use of the \mmt\ SCSCP client to communicate with an external SCSCP server. 
    In this case, the SCSCP server communicates with a server provided by the GAP system.  
  }
  \label{fig:scscpclient}
\end{figure}