% !TEX root = ../thesis.tex
\begin{figure}[h]
  \begin{center}
    \begin{tikzpicture}[xscale=3.9,yscale=2.2]\footnotesize
      \node[thy] (sg) at (1,-1) {
        \begin{tabular}{l}
          \textsf{Semigroup}\\\hline
          $G,\circ$\\\hline
          $\scriptstyle x\circ y\in G$ \\\hline
          $\scriptstyle \plaintt{assoc}: (x\circ y)\circ z=x\circ (y\circ z)$
        \end{tabular}
      };
      \node[thy] (m) at (1,0) {
        \begin{tabular}{l}
          \textsf{Monoid}\\\hline
          $e$\\\hline
          $\scriptstyle e\circ x=x$
        \end{tabular}
      };
      \node[thy] (g) at (1,1) {
        \begin{tabular}{l}
          \textsf{Group}\\\hline
          $i := \scriptstyle\lambda x.\tau y. x\circ y=e$\\\hline
          $\scriptstyle\forall x:G.\exists y:G.x\circ y=e$
        \end{tabular}
      };
      \node[thy] (cg) at (1,2.3) {
        \begin{tabular}{l}
          \textsf{Commutative Group}\\\hline
          \\\hline
          $\scriptstyle \plaintt{comm} : x\circ y=y\circ x$
        \end{tabular}
      };

      \node[thy] (N) at (-1,-1) {
        \begin{tabular}{l}
          \textsf{Natural Numbers}\\\hline
          $\mathbb{N},s,0$\\\hline
          $\scriptstyle P1$,\ldots $\scriptstyle P5$
        \end{tabular}
      };
      \node[thy] (Np) at (-1,0) {
        \begin{tabular}{l}
          \textsf{Natural Numbers with Plus}\\\hline
          $+$\\\hline
          $\scriptstyle n+0=n$,\\
          $\scriptstyle n+s(m)=s(n+m)$
        \end{tabular}
      };
      \node[thy] (Nt) at (-1,1) {
        \begin{tabular}{l}
          \textsf{Natural Numbers with Multiplication}\\\hline
          $\cdot$\\\hline
          $\scriptstyle n\cdot1=n$,\\
          $\scriptstyle n\cdot s(m)=n\cdot m+n$
        \end{tabular}
      };
      \node[thy] (ia) at (-1,2.3) {

        \begin{tabular}{l}
          \textsf{Integer Arithmetics}\\\hline
          $-$\\
          $\mathbb{Z} := \mathbb{N}\cup-\mathbb{N}$\\\hline
          $\scriptstyle-0=0$
        \end{tabular}
      };

      %Right side
      \draw[include] (sg) -- (m);
      \draw[include] (m) -- node[left] (mg) {$\mathsf{g}$} (g);
      \draw[include] (g) -- (cg);

      %Left side
      \draw[include] (N) -- (Np);
      \draw[include] (Np) -- (Nt);
      
      \draw[struct] (Nt) to[bend left=10] node[left] {$p$} (ia);
      \draw[struct] (Nt) to[bend right=10] node[right] {$n$} (ia);  

      %Links
      \draw[view] (m) -- node[above] {$
      \thmo{f}{
      \left\{\begin{array}{l}
        G\mapsto\mathbb{N}\\
        \circ\mapsto +\\
        e\mapsto 0
      \end{array}\right\}
      }$
      } (Np);
      \draw[view] (cg) -- node[above] {$\thmo{h}{
      \left\{\begin{array}{l}
        i\mapsto -\\
        \mathsf{g}\mapsto\mathsf{f}
      \end{array}\right\}
      }$} (ia);
    \end{tikzpicture}
  \end{center}

  \caption[A Simple Theory Graph]{
  Simplified example of a theory graph. 
  Imports and Structures are represented as solid edges and views as wavy edges. 
  The mappings of the views are given explicitly. 
  We omit the complete definitions of the Peano axioms for the natural numbers.
  }
  \label{fig:theorygraph}
\end{figure}